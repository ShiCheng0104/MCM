\newpage
\section{Notations}
\setlength{\parindent}{2em} % 设置缩进为2个字符宽度


The core symbols and their definitions used in this study are summarized in Table~\ref{table:notations}, providing an overview of the key parameters and their related meanings.

\begin{table}[ht]
\centering
\caption{Notations used in this literature}
\renewcommand{\arraystretch}{1.5} % Adjust row spacing
\begin{tabular}{>{\centering\arraybackslash}p{0.3\linewidth} >{\centering\arraybackslash}p{0.6\linewidth}} % Two columns with width ratio, centered alignment
\toprule % Top line
\textbf{Symbol} & \textbf{Description} \\ 
\midrule % Middle line

$n$ & 第$w$周剩余选手数量 \\

$S_i$ & 选手$i$的评委总分 \\

$V_i$ & 选手$i$的估计观众投票数 \\

$E_i$ & 淘汰指示器:若选手$i$被淘汰则$E_i = 1$,否则为0 \\

$\mathcal{M}$ & 投票组合方法:``rank''(排名法)或``percent''(百分比法) \\

$C_i$ & 选手$i$在方法$\mathcal{M}$下的综合得分 \\

$\alpha$ & 投票与评委得分关系的幂律指数 \\

$\boldsymbol{\theta}$ & 先验偏好因子(舞伴效应、行业效应) \\

$R_i^{\text{judge}}$ & 选手$i$的评委得分排名 \\

$R_i^{\text{fan}}$ & 选手$i$的观众投票排名 \\

$\beta_p$ & 专业舞伴$p$的投票偏好效应 \\

$\gamma_d$ & 名人行业$d$的投票偏好效应 \\

$B_i^w$ & 选手$i$在第$w$周的存活提升值 \\

$\mathbf{v}$ & 观众投票份额向量 $(v_1, \ldots, v_n)$ \\


\bottomrule % Bottom line
\end{tabular}
\label{table:notations}
\end{table}
%%%%%%%%%%%%%%%%%%%%%%%%%%%%%%%%%%%%%%%%%%%%%%%%%%%%%%%%%%%%%%%%%%%%%%%%%
%%%%%%%%%%%%%%%%%%%%%%%%%%%%%%%%%%%%%%%%%%%%%%%%%%%%%%%%%%%%%%%%%%%%%%%%%
%%%%%%%%%%%%%%%%%%%%%%%%%%%%%%%%%%%%%%%%%%%%%%%%%%%%%%%%%%%%%%%%%%%%%%%%%
%%%%%%%%%%%%%%%%%%%%%%%%%%%%%%%%%%%%%%%%%%%%%%%%%%%%%%%%%%%%%%%%%%%%%%%%%
%%%%%%%%%%%%%%%%%%%%%%%%%%%%%%%%%%%%%%%%%%%%%%%%%%%%%%%%%%%%%%%%%%%%%%%%%



