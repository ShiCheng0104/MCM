\newpage
\section{问题1:观众投票估计模型}

\subsection{问题剖析与建模策略}

问题1本质上是一个\textbf{欠定逆问题}(underdetermined inverse problem):在观众投票这一隐变量缺失的情况下,从淘汰结果和评委得分的联合观测中进行统计推断。这类问题在计量经济学和信号处理领域广泛存在,其核心挑战在于解的非唯一性与模型的可识别性(identifiability)。

\textbf{问题的数学结构}。设第$w$周有$n$名选手,评委得分$\mathbf{S} = (S_1, \ldots, S_n)$可观测,观众投票$\mathbf{V} = (V_1, \ldots, V_n)$不可观测,淘汰结果$E$已知。问题可形式化为:给定映射$f: (\mathbf{S}, \mathbf{V}) \rightarrow E$和观测$({\mathbf{S}, E})$,反推$\mathbf{V}$。由于$|\mathbf{V}| = n$个未知数仅对应1个淘汰约束,系统严重欠定($n \gg 1$)。

\textbf{可识别性困境}。多个投票分布可产生相同淘汰结果,导致解不唯一。传统方法有二:(1)施加正则化约束(如最大熵、稀疏性),但缺乏投票行为的先验支持;(2)引入辅助假设(如投票与得分的参数关系),但可能过度简化现实。我们的策略是将两者结合——用行为模型提供软约束,用淘汰结果施加硬约束,在约束空间中搜索最符合经验规律的解。

\textbf{方法演变的影响}。第28季前后投票组合规则的变化(排名法$\leftrightarrow$百分比法)改变了$f$的函数形式,使得相同的$(\mathbf{S}, \mathbf{V})$可能导致不同的$E$。这要求模型具备\textit{规则自适应性},能够在统一框架下处理两种机制。

基于上述分析,我们设计\textbf{双层贝叶斯框架}:基线层建立$P(\mathbf{V}|\mathbf{S})$的先验分布,反推层通过$P(\mathbf{V}|\mathbf{S}, E)$进行后验推断。这种分层设计既保证了理论严谨性(后验一致性),又具备实践可行性(先验可解释性)。

\subsection{投票组合方法}

不同赛季使用的两种投票组合方法为(相关符号定义见表~\ref{table:notations}):

\textbf{排名法}(赛季1-2、28-34):
\begin{equation}
C_i^{\text{rank}} = R_i^{\text{judge}} + R_i^{\text{fan}}
\end{equation}
其中$R_i^{\text{judge}} = \text{rank}(-S_i)$和$R_i^{\text{fan}} = \text{rank}(-V_i)$。综合排名\textit{最高}(即$C_i^{\text{rank}}$最大,表现最差)的选手被淘汰。

\textbf{百分比法}(赛季3-27):
\begin{equation}
C_i^{\text{percent}} = \frac{S_i}{\sum_{j=1}^n S_j} + \frac{V_i}{\sum_{j=1}^n V_j}
\end{equation}
综合百分比\textit{最低}的选手被淘汰。

\subsection{基线模型:幂律投票假设}

观众投票与评委得分的关系既非完全独立,亦非严格线性。我们提出\textbf{幂律假设}捕捉这种非线性依赖:
\begin{equation}
V_i \propto S_i^\alpha, \quad \hat{V}_i = V_{\text{total}} \cdot \frac{S_i^\alpha}{\sum_{j=1}^n S_j^\alpha}
\end{equation}

指数$\alpha$的经济学解释:当$\alpha > 1$时,高分选手获得超比例支持("马太效应");$\alpha < 1$时体现"同情弱者";$\alpha = 1$退化为比例模型。通过网格搜索最大化淘汰预测准确率,校准得$\alpha^* \approx 1.2$,印证观众存在温和的"强者偏好"。

不确定性通过蒙特卡洛采样量化:向幂律预测注入15\%高斯噪声,生成100个样本构建95\%置信区间。这为后续分析提供了统计显著性检验的基础。

\subsection{精确反推模型:约束优化框架}

基线模型提供先验分布$P(\mathbf{V}|\mathbf{S})$,但无法保证后验一致性。我们建立约束优化框架,从淘汰结果反推投票的后验分布$P(\mathbf{V}|\mathbf{S}, E)$。

\subsubsection{投票偏好因子的学习}

在优化前,从全局数据学习系统性偏好。定义"存活提升"$B_i^w$衡量评分排名与生存结果的错位程度:
\begin{equation}
B_i^w = \begin{cases}
-(n - R_i^{\text{score}})/n & \text{若被淘汰(评分越高越意外)} \\
(R_i^{\text{score}} - 1)/n & \text{若存活(评分越低越意外)}
\end{cases}
\end{equation}
其中$R_i^{\text{score}}$为选手$i$的评分排名(1=最高分)。

按专业舞伴$p$和名人行业$d$聚合提升值,得效应参数:
\begin{align}
\beta_p &= \frac{1}{|\mathcal{W}_p|} \sum_{w \in \mathcal{W}_p} B_i^w - \mu_{\text{overall}} \\
\gamma_d &= \frac{1}{|\mathcal{W}_d|} \sum_{w \in \mathcal{W}_d} B_i^w - \mu_{\text{overall}}
\end{align}

结合标准化年龄,构建先验偏好因子:
\begin{equation}
\theta_i = \beta_{p(i)} + \gamma_{d(i)} + \delta_{\text{age}} \cdot \frac{\text{age}_i - \mu_{\text{age}}}{\sigma_{\text{age}}}
\end{equation}

这些偏好因子识别了哪些舞伴、行业和年龄段的选手系统性地获得更多或更少的观众支持。正效应表示该因素带来额外投票支持,负效应则相反。

\subsubsection{优化问题的数学构建}

对投票份额$\mathbf{v} = (v_1, \ldots, v_n)$(满足$v_i \in [0,1], \sum v_i = 1$),最小化目标函数:
\begin{equation}
\mathcal{J}(\mathbf{v}) = \left\| \mathbf{v} - \mathbf{v}^{\text{prior}} \right\|_2^2 + \lambda \cdot \mathcal{L}_{\text{elim}}(\mathbf{v})
\end{equation}
其中先验份额为:
\begin{equation}
\mathbf{v}^{\text{prior}} = \text{normalize}\left[\frac{\mathbf{S}}{\sum_j S_j} \odot (1 + \boldsymbol{\theta})\right]
\end{equation}

\textbf{淘汰约束的软化处理}。对于排名法,离散排名函数不可微,我们采用\textbf{sigmoid软排名近似}:
\begin{equation}
\text{SoftRank}_i = n - \sum_{j \neq i} \frac{1}{1 + \exp[(x_j - x_i)/T]}
\end{equation}
其中$x$可以是分数或投票份额,$T$为温度参数(代码中设为0.1)。该函数近似计算"有多少人比$i$差",温度越小越接近真实排名。

综合软排名为$C_i^{\text{soft}} = \text{SoftRank}_i(\mathbf{S}) + \text{SoftRank}_i(\mathbf{v})$。淘汰约束惩罚项为:
\begin{equation}
\mathcal{L}_{\text{rank}}(\mathbf{v}) = \sum_{\substack{i: E_i=1 \\ j: E_j=0}} \max(0, C_j^{\text{soft}} - C_i^{\text{soft}} + \delta)^2
\end{equation}
其中边际$\delta = 0.5$确保被淘汰者的软排名显著高于存活者。

对于百分比法,约束为:
\begin{equation}
\mathcal{L}_{\text{percent}}(\mathbf{v}) = \sum_{\substack{i: E_i=1 \\ j: E_j=0}} \max(0, C_i - C_j + \epsilon)^2
\end{equation}
其中$C_i = S_i/\sum S_j + v_i$,边际$\epsilon = 0.01$。

\textbf{正则化项的作用}。$\|\mathbf{v} - \mathbf{v}^{\text{prior}}\|_2^2$防止过拟合,在众多满足约束的解中选择最接近先验的那个。权重$\lambda = 1000$通过实验确定,平衡正则化与约束满足。

\subsubsection{SLSQP算法求解}

采用序列最小二乘规划(SLSQP)求解约束优化。该算法适合处理等式约束$\sum v_i = 1$和盒约束$v_i \in [0.001, 0.999]$,基于梯度信息快速收敛。

初始值设为先验$\mathbf{v}^{\text{prior}}$,优化参数为:最大迭代500次,收敛容差$10^{-8}$。成功收敛的判据是目标函数值低于阈值且淘汰约束得到满足。

\textbf{失败案例分析}。对于约束无法满足的周次(优化失败),模型回退到备用估计:
\begin{equation}
\mathbf{v}^{\text{fallback}} = \text{normalize}\left[\frac{\mathbf{S}}{\sum S_j} \odot (1 + 0.3 \cdot \boldsymbol{\theta})\right]
\end{equation}
这些失败案例往往对应评分与淘汰结果存在强矛盾(如高分选手被淘汰),或数据记录异常(多人同时淘汰/无人淘汰)。

\subsection{模型评估与集成策略}

\subsubsection{一致性度量体系}

评估投票估计需要多个互补指标。代码中实现了以下度量:

\textbf{精确匹配准确率}定义为:
\begin{equation}
\text{ACC}_{\text{exact}} = \frac{1}{W} \sum_{w=1}^W \mathbb{1}\left\{ \text{predicted}(w) = \text{actual}(w) \right\}
\end{equation}
衡量能否精准预测每周被淘汰者,这是最严格的标准。

\textbf{底部N准确率}放宽判定条件,检验实际淘汰者是否落入预测的"危险区"(综合得分最低的若干名)。这更符合实际应用——关心"谁有风险"而非"谁一定淘汰"。

\textbf{约束满足边际}量化淘汰的"确定性":
\begin{equation}
\text{Margin}_k = \min_{i: E_i = 0} |C_k - C_i|
\end{equation}
边际大表示淘汰显著,边际小则属"惊险淘汰",投票的微小波动即可改变结果。

\subsubsection{集成策略的实现}

代码采用场景感知的模型选择机制:

\textbf{淘汰周次}:优先使用精确反推模型。运行SLSQP优化,若收敛则采用优化结果。

\textbf{非淘汰周次}:使用基线模型结合偏好因子。具体为:
\begin{equation}
\mathbf{v}^{\text{non-elim}} = 0.7 \cdot \mathbf{v}^{\text{baseline}} + 0.3 \cdot \mathbf{v}^{\text{prior}}
\end{equation}
其中$\mathbf{v}^{\text{baseline}}$为基线模型的幂律估计,$\mathbf{v}^{\text{prior}}$为带偏好因子的先验估计。混合比例70:30平衡了统计规律与个体特征。

\textbf{优化失败周次}:回退到备用估计方法。使用评分比例乘以偏好调整因子(权重0.3),确保即使约束无法满足也能提供合理估计。

\subsubsection{不确定性的传播}

对于基线模型,不确定性通过蒙特卡洛采样量化。生成100个带噪声样本(噪声水平15\%),构建95\%置信区间$[\text{Q}_{2.5\%}, \text{Q}_{97.5\%}]$。

对于精确反推模型,不确定性体现在约束满足边际上。边际越小,说明多个投票分布都可能产生相同淘汰结果,估计的唯一性越差。

代码中还实现了标准化残差分析:
\begin{equation}
r_i = \frac{v_i^* - v_i^{\text{prior}}}{\sigma_i^{\text{prior}}}
\end{equation}
其中$v_i^*$为优化得到的估计,$\sigma_i^{\text{prior}}$为先验标准差(通过蒙特卡洛估计)。$|r_i| > 2$被标记为异常,提示该选手的实际投票显著偏离预期。

\subsubsection{双视角可解释性}

集成策略提供理解投票的两个维度:

\begin{itemize}
    \item \textit{基线视角}:揭示"常规投票模式"——若观众按评分、舞伴、行业等因素理性投票,结果应该如何。
    \item \textit{精确视角}:揭示"实际投票需求"——为了产生观察到的淘汰,投票实际上必须是怎样的。
\end{itemize}

两者的偏离$\Delta_i = v_i^* - v_i^{\text{baseline}}$量化"异常程度"。$\Delta$大的周次往往对应争议事件(如人气选手意外被淘汰,或低分选手意外晋级)。这种差异分析为理解节目的争议性结果(如题目提到的Bobby Bones、Jerry Rice等案例)提供了定量工具。

\section{问题2:投票方法对比与推荐}

利用问题1估计的观众投票数据,本节对两种投票组合方法进行系统对比。问题2包含三个子问题,我们将逐一清晰回答。

\subsection{子问题2.1:两种方法的结果对比与观众投票偏向性}

\subsubsection{问题陈述}

\textit{对所有赛季应用两种方法(排名法和百分比法),比较结果差异。如果存在差异,哪种方法更偏向观众投票?}

\subsubsection{分析方法}

我们采用\textbf{反事实模拟}:对每个历史周次,使用估计的观众投票,分别计算两种方法下的淘汰结果。

\textbf{排名法}:
\begin{equation}
C_i^{\text{rank}} = \text{rank}(-S_i) + \text{rank}(-\hat{V}_i), \quad k^{\text{rank}} = \arg\max_i C_i^{\text{rank}}
\end{equation}

\textbf{百分比法}:
\begin{equation}
C_i^{\text{percent}} = \frac{S_i}{\sum_j S_j} + \frac{\hat{V}_i}{\sum_j \hat{V}_j}, \quad k^{\text{percent}} = \arg\min_i C_i^{\text{percent}}
\end{equation}

对34个赛季、335个周次进行完整模拟,统计$k^{\text{rank}} \neq k^{\text{percent}}$的频率。

\subsubsection{结果1:方法一致性分析}

\begin{table}[h]
\centering
\begin{tabular}{lcc}
\hline
\textbf{统计指标} & \textbf{数值} & \textbf{百分比} \\
\hline
总分析周次 & 335 & 100\% \\
结果一致周次 & 251 & 74.9\% \\
结果不一致周次 & 84 & 25.1\% \\
\hline
\end{tabular}
\caption{排名法与百分比法的结果一致性}
\label{tab:method_agreement}
\end{table}

\textbf{关键发现}:25.1\%的不一致率表明投票方法的选择对比赛结果有\textbf{实质性影响}。这不是微小的技术差异,而是可能改变约四分之一选手命运的系统性因素。

\subsubsection{结果2:观众投票偏向性分析}

在84个不一致周次中,我们定义:若方法$A$淘汰的选手评委得分\textit{高于}方法$B$淘汰的选手,则方法$A$更"偏向观众"(因为它牺牲了高评委分选手以保留高票选手)。

\begin{table}[h]
\centering
\begin{tabular}{lcc}
\hline
\textbf{偏向类型} & \textbf{案例数} & \textbf{占比} \\
\hline
排名法更偏向观众 & 16 & 19.0\% \\
百分比法更偏向观众 & 68 & 81.0\% \\
\hline
\end{tabular}
\caption{两种方法的观众投票偏向性对比}
\label{tab:fan_bias}
\end{table}

\textbf{机制解释}:排名法将得分/投票映射为离散排名,抹去数值差距信息。例如评委打分差10分和差1分,在排名中都可能仅体现为"相邻一名"。这种\textbf{非线性压缩}削弱了观众投票的数值优势。

相反,百分比法保留数值信息:
\begin{equation}
\Delta C_i^{\text{percent}} = \frac{S_i - S_j}{\sum S_k} + \frac{V_i - V_j}{\sum V_k}
\end{equation}
观众投票的优势能\textbf{线性地}弥补评委得分劣势。

\subsubsection{子问题2.1的明确回答}

\noindent\fcolorbox{black}{gray!10}{%
\begin{minipage}{0.95\textwidth}
\vspace{0.3em}
\noindent\textbf{子问题2.1回答}

\textbf{Q1: 两种方法结果是否有差异?}

\textbf{答}:是的。在335个分析周次中,84周次(25.1\%)的淘汰结果不同。

\textbf{Q2: 哪种方法更偏向观众投票?}

\textbf{答}:百分比法显著更偏向观众投票。在不一致案例中,百分比法偏向观众的比例为81.0\%,排名法仅19.0\%($p < 0.001$)。原因是百分比法的线性加权更充分地体现了观众投票的数值优势,而排名法的离散化削弱了这种效应。
\vspace{0.3em}
\end{minipage}%
}

\subsection{子问题2.2:争议性选手案例分析}

\subsubsection{问题陈述}

\textit{对于存在评委-观众意见分歧的争议性选手(Jerry Rice、Billy Ray Cyrus、Bristol Palin、Bobby Bones),投票方法的选择是否会改变结果?评委裁决机制(从底部两人中选择)的影响如何?}

\subsubsection{四位争议选手的数据总结}

\begin{table}[h]
\centering
\small
\begin{tabular}{lcccccc}
\hline
\textbf{选手} & \textbf{赛季} & \textbf{实际名次} & \textbf{比赛周数} & \textbf{最低分周数} & \textbf{排名法底2次数} & \textbf{百分比法底2次数} \\
\hline
Jerry Rice & 2 & 2 & 8 & 2 & 3 & 2 \\
Billy Ray Cyrus & 4 & 5 & 8 & 3 & 4 & 4 \\
Bristol Palin & 11 & 3 & 10 & 5 & 8 & 7 \\
Bobby Bones & 27 & 1 & 9 & 2 & 3 & 2 \\
\hline
\end{tabular}
\caption{争议性选手的比赛数据统计}
\label{tab:controversy_summary}
\end{table}

\subsubsection{案例1:Bristol Palin - 方法影响最显著}

Bristol Palin(第11季,最终第3名)在10周比赛中5次获评委最低分,却最终获得季军,引发争议。

\textbf{排名法模拟}:她在10周中有8周位于底部两名,多次面临淘汰风险。评委得分排名持续垫底(平均第9-10名),即使观众投票排名中等(第5-6名),综合排名$C^{\text{rank}}$仍经常最差。

\textbf{百分比法模拟}:10周中7周位于底部两名,但频率略低。她的观众投票份额虽非最高,但\textit{显著高于}评委得分比例。百分比法的线性加权使这种优势累积,延长生存期。

\textbf{结论}:Bristol Palin是唯一在两种方法下表现\textbf{明显不同}的案例。实际她获第3名,证实了百分比法(当时采用的方法)确实让她受益。这印证了我们关于百分比法更偏向观众的结论。

\subsubsection{案例2、3、4:Jerry Rice、Billy Ray Cyrus、Bobby Bones}

这三位选手的共同特征:
\begin{itemize}
    \item \textbf{观众投票优势极其巨大},足以在任何合理组合方法下抵消评委劣势
    \item 两种方法的模拟结果\textbf{基本一致}——都能存活到决赛阶段
    \item 争议的本质不在于投票方法,而在于\textbf{评委与观众意见的根本分歧}
\end{itemize}

例如Bobby Bones(第27季冠军),虽然评委得分不高,但观众投票份额估计达到30-35\%(远高于平均的11-12\%)。这种压倒性优势使得无论采用排名法还是百分比法,他都能稳定晋级。

\subsubsection{评委裁决机制的影响分析}

第28季引入的机制:用综合得分确定底部两名$\rightarrow$评委投票决定淘汰谁。

\textbf{模拟假设}:评委在底部两人中总是淘汰评委得分更低者(维护技术标准)。

\textbf{影响估算}:分析全部335周次,若实施评委裁决:
\begin{itemize}
    \item 约43.5\%的周次中,原方法淘汰的选手评委得分\textit{高于}另一位底部选手
    \item 在这些情况下,评委裁决会\textbf{改变结果}
    \item 估算的结果改变率:$0.435 \times 0.442 = 19.2\%$
\end{itemize}

\textbf{对争议选手的影响}:
\begin{itemize}
    \item \textbf{Bristol Palin}:8次底部两名中,至少5次她是评委分更低的那位,评委裁决会让她更早出局
    \item \textbf{其他三位}:同样会因频繁位于底部而多次面临评委否决,生存难度显著增加
\end{itemize}

\subsubsection{子问题2.2的明确回答}

\noindent\fcolorbox{black}{gray!10}{%
\begin{minipage}{0.95\textwidth}
\vspace{0.3em}
\noindent\textbf{子问题2.2回答}

\textbf{Q1: 投票方法选择会改变争议选手的结果吗?}

\textbf{答}:\textit{部分会,部分不会}。
\begin{itemize}
    \item \textbf{Bristol Palin}:会受影响。百分比法让她更容易存活(7次底部vs.8次),实际她在百分比法下获第3名。
    \item \textbf{Jerry Rice、Billy Ray Cyrus、Bobby Bones}:基本不受影响。他们的观众投票优势过于巨大,在两种方法下都能进入决赛。
\end{itemize}

\textbf{Q2: 评委裁决机制的影响如何?}

\textbf{答}:\textbf{显著且实质性}。
\begin{itemize}
    \item 预计改变约19.2\%周次的结果
    \item 对四位争议选手都会产生\textbf{不利影响}——他们频繁位于底部,评委倾向于淘汰低分者,会让他们更早出局
    \item 该机制是对"观众主导"的\textbf{制衡},能有效防止评委意见被完全忽视
\end{itemize}
\vspace{0.3em}
\end{minipage}%
}

\subsection{子问题2.3:方法推荐与理由}

\subsubsection{问题陈述}

\textit{基于分析,推荐使用哪种方法?是否建议加入评委裁决机制?}

\subsubsection{推荐方案与理由}

我们推荐:\textbf{百分比法 + 评委裁决机制}

\textbf{选择百分比法的理由}:(1) \textbf{信息保留}:保留数值差距,排名法的离散化会丢失关键信息;(2) \textbf{观众友好}:数据显示百分比法更偏向观众(81.0\%),符合娱乐节目定位;(3) \textbf{可预测性}:线性结构便于理解投票与结果的关系。

\textbf{保留评委裁决的理由}:(1) \textbf{质量底线}:防止技术水平过低的选手仅凭人气晋级,维护节目专业性;(2) \textbf{适度制衡}:影响约19\%周次,平衡观众意愿与技术标准;(3) \textbf{透明实施}:评委裁决应公开投票理由,避免争议。

\subsubsection{子问题2.3的明确回答}

\noindent\fcolorbox{black}{gray!10}{%
\begin{minipage}{0.95\textwidth}
\vspace{0.3em}
\noindent\textbf{子问题2.3回答}

\textbf{Q1: 推荐使用哪种投票方法?}

\textbf{答}:\textbf{百分比法}。

\textbf{三大理由}:
\begin{enumerate}
    \item \textbf{信息保留}:保留数值差距,更精确反映实际水平
    \item \textbf{观众友好}:更偏向观众投票(81.0\%),符合娱乐节目定位
    \item \textbf{可预测性}:线性结构易于理解,增强互动性
\end{enumerate}

\textbf{Q2: 是否建议加入评委裁决机制?}

\textbf{答}:\textbf{是的,建议保留}。

\textbf{三大理由}:
\begin{enumerate}
    \item \textbf{质量底线}:防止零技术全人气的极端情况
    \item \textbf{适度制衡}:影响约19\%结果,平衡专业性与娱乐性
    \item \textbf{戏剧性}:增加悬念和可看性
\end{enumerate}

\textbf{实施建议}:评委裁决应透明化,公开投票理由,避免黑箱质疑。
\vspace{0.3em}
\end{minipage}%
}



